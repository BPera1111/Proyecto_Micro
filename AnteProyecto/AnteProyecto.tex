% \documentclass{article}

% \usepackage{graphicx} % Paquete para incluir imágenes
% \usepackage[utf8]{inputenc} % Para caracteres especiales
% \usepackage[spanish]\begin{itemize}
%   \item \textbf{Mapa de pines Blue Pill STM32F103C8T6}: Ver referencia [6]
%   \item \textbf{Esquema RAMPS 1.4}: Ver referencia [7]
% \end{itemize}el} % Para idioma español
% \usepackage{xcolor} % Para colores
% \usepackage{amsmath} % Para ecuaciones
% \usepackage{amssymb} % Para símbolos matemáticos adicionales (incluye \therefore)
% \usepackage{fancyhdr} % Para encabezados y pies de página
% %\usepackage{geometry} % Para ajustar márgenes
% \usepackage{caption} % Para usar \captionof fuera de float
% \usepackage{float}
% \usepackage{multirow}
% \usepackage{enumitem}
% \usepackage{lastpage}% Para el número de la última página
% \usepackage{hyperref}
% \hypersetup{
%     colorlinks=true,
%     linkcolor=black,
%     citecolor=black,
%     urlcolor=black,
%     pdfborder={0 0 0}
% }

% % Configuración de márgenes
% \usepackage[a4paper,
%   top=1.9cm,
%   bottom=3.67cm,
%   left=1.9cm,
%   right=1.32cm
% ]{geometry}


% % Configuración de encabezados y pies de página
% \pagestyle{fancy}
% \fancyhf{} % Limpia encabezados y pies de página

% % Encabezado
% \fancyhead[L]{\scriptsize UNCuyo - Ing. Mecatrónica\\Mendoza - Argentina}
% \fancyhead[C]{\scriptsize\textbf{Espacio Curricular:  \textbf{Microcontroladores y Electrónica de Potencia (Año 2024)}\\
% \textbf{PROYECTO GLOBAL INTEGRADOR}}}
% \fancyhead[R]{\scriptsize Peña Lautaro, Peralta Bruno\\19/07/2025}

% % Pie de página
% \fancyfoot[C]{Pág. \thepage\ de \pageref{LastPage}}

% \setlength{\headheight}{16.9528pt}  % importante si usás fancyhdr con varios renglones
% \setlength{\headsep}{0.5cm}    % separación entre cabecera y texto

% % \fancyhead[L]{Automática y Máquinas Eléctricas} % Encabezado izquierdo
% % \fancyhead[C]{2025} % Encabezado central
% % \fancyhead[R]{Peña Lautaro, Peralta Bruno} % Encabezado derecho
% % \fancyfoot[C]{\thepage} % Número de página centrado en el pie de página

% \begin{document}

% % Portada
% \begin{titlepage}
%     \centering

%     % Logo
%     \includegraphics[width=0.8\textwidth]{img/fing.jpeg}\par\vspace{1cm}
%     \vspace{2cm}
%     % Título
%     {\Huge \textbf{Proyecto Global Integrador:}\\[0.5cm]
%     \textbf{CNC Laser de 3 Ejes}\vspace{2cm}}
%     % \textbf{Motor Sincrónico de Imanes Permanentes}\par}\vspace{2cm}
    
%     % Autores
%     {\Large Peña Lautaro - 13099\\
%     Peralta Bruno - 13220\par}\vspace{0.5cm}
    
%     % Profesor titular
%     {Año 2024\par}
% \end{titlepage}

% % Índice
% \newpage
% \tableofcontents
% \newpage





% \end{document}

\documentclass[12pt]{article}
\usepackage[spanish]{babel}
\usepackage[utf8]{inputenc}
\usepackage{graphicx}
\usepackage{amsmath}
\usepackage{xcolor}
\usepackage{tikz}
\usetikzlibrary{arrows.meta, positioning}
\usepackage{url}
\usepackage{hyperref}
\hypersetup{
    colorlinks=true,
    linkcolor=black,
    citecolor=black,
    urlcolor=black,
    pdfborder={0 0 0}
}
\usepackage{geometry}
\usepackage{fancyhdr}
\usepackage{lastpage}
\geometry{a4paper, margin=1.3cm}

\title{Anteproyecto: Impresora 3D Modificada para Grabado Láser CNC de PCB}
\author{Nombre del estudiante}
\date{\today}



% Configuración de encabezados y pies de página
\pagestyle{fancy}
\fancyhf{} % Limpia encabezados y pies de página

% Encabezado
\fancyhead[L]{\scriptsize UNCuyo - Ing. Mecatrónica\\Mendoza - Argentina}
\fancyhead[C]{\scriptsize\textbf{\textbf{Microcontroladores y Electrónica de Potencia (Año 2024)}\\
\textbf{PROYECTO GLOBAL INTEGRADOR}}}
\fancyhead[R]{\scriptsize Peña Lautaro, Peralta Bruno\\19/07/2025}

% Pie de página
\fancyfoot[C]{Pág. \thepage\ de \pageref{LastPage}}

\setlength{\headheight}{16.9528pt}  % importante si usás fancyhdr con varios renglones
\setlength{\headsep}{0.5cm}    % separación entre cabecera y texto

% \fancyhead[L]{Automática y Máquinas Eléctricas} % Encabezado izquierdo
% \fancyhead[C]{2025} % Encabezado central
% \fancyhead[R]{Peña Lautaro, Peralta Bruno} % Encabezado derecho
% \fancyfoot[C]{\thepage} % Número de página centrado en el pie de página

\begin{document}

% Portada
\begin{titlepage}
    \centering

    % Logo
    \includegraphics[width=0.8\textwidth]{img/fing.jpeg}\par\vspace{1cm}
    \vspace{2cm}
    % Título
    {\Huge \textbf{Proyecto Global Integrador:}\\[0.5cm]
    \textbf{CNC Laser de 3 Ejes}\vspace{2cm}}
    % \textbf{Motor Sincrónico de Imanes Permanentes}\par}\vspace{2cm}
    
    % Autores
    {\Large Peña Lautaro - 13099\\
    Peralta Bruno - 13220\par}\vspace{0.5cm}
    
    % Profesor titular
    {Año 2024\par}
\end{titlepage}

% Índice
\newpage
\tableofcontents
\newpage

\section{Introducción}

El presente proyecto consiste en la modificación de una impresora 3D antigua con el fin de reutilizar su estructura y componentes para desarrollar una máquina CNC de grabado láser destinada a la creación rápida de placas de circuito impreso (PCB). Esta herramienta permitirá fabricar prototipos de forma ágil para proyectos de electrónica y robótica.

La motivación principal radica en el desafío de aprovechar un equipo en desuso y otorgarle una nueva funcionalidad, aportando una solución económica y eficiente para la etapa de prototipado de circuitos. El sistema combina conocimientos de control de motores, programación de microcontroladores, comunicación serial y ejecución de comandos G-code.

\section{Esquema Tecnológico}

\begin{center}
\begin{tikzpicture}[node distance=3cm, auto]
  % Bloques principales
  \node[draw, rectangle, rounded corners, fill=blue!10] (pc) {PC con G-code};
  \node[draw, rectangle, rounded corners, fill=green!10, below of=pc, yshift=-0.5cm] (bluepill) {Blue Pill STM32F103C8T6};
  \node[draw, rectangle, rounded corners, fill=yellow!20, below of=bluepill, yshift=-0.5cm] (ramp) {RAMPS 1.4};
  \node[draw, rectangle, rounded corners, fill=orange!10, below of=ramp, xshift=-5.5cm, yshift=-0.5cm] (driverX) {Driver A4988 - Eje X};
  \node[draw, rectangle, rounded corners, fill=orange!10, below of=ramp, xshift=0cm, yshift=-0.5cm] (driverY) {Driver A4988 - Eje Y};
  \node[draw, rectangle, rounded corners, fill=orange!10, below of=ramp, xshift=5.5cm, yshift=-0.5cm] (driverZ) {Driver A4988 - Eje Z};
  \node[draw, rectangle, rounded corners, fill=red!10, below of=driverX, yshift=-0.5cm] (motorX) {M.PAP NEMA 17HS3430};
  \node[draw, rectangle, rounded corners, fill=red!10, below of=driverY, yshift=-0.5cm] (motorY) {M.PAP NEMA 17HS8401};
  \node[draw, rectangle, rounded corners, fill=red!10, below of=driverZ, yshift=-0.5cm] (motorZ) {M.PAP NEMA 17HS8401};
  \node[draw, rectangle, rounded corners, fill=gray!20, right of=pc, xshift=5cm] (serial) {Monitor Serial};
  \node[draw, rectangle, rounded corners, fill=cyan!20, right of=ramp, xshift=6cm] (endstops) {Fines de carrera};

  % Conexiones
  \draw[-{Stealth}] (pc) -- (bluepill) node[midway, left] {USB Serial};
  \draw[-{Stealth}] (bluepill) -- (ramp) node[midway, left] {Interfaz};
  \draw[-{Stealth}] (ramp) -- (driverX);
  \draw[-{Stealth}] (ramp) -- (driverY);
  \draw[-{Stealth}] (ramp) -- (driverZ);
  \draw[-{Stealth}] (driverX) -- (motorX);
  \draw[-{Stealth}] (driverY) -- (motorY);
  \draw[-{Stealth}] (driverZ) -- (motorZ);
  \draw[-{Stealth}, dashed] (ramp) -- (endstops) node[midway, above] {Sensores};
  \draw[-{Stealth}, dashed] (pc) -- (serial);
\end{tikzpicture}
\end{center}

\section{Detalle de Módulos}

\subsection{Microcontrolador: Blue Pill STM32F103C8T6}
Microcontrolador de 32 bits basado en ARM Cortex-M3, con 72 MHz, 64KB Flash y 20KB RAM. Opera a 3.3V, pero varios pines son tolerantes a 5V. Se programa mediante STM32CubeIDE. Más información en referencia \hyperref[ref1]{[1]}.

\subsection{Drivers A4988}
Módulos controladores de motores paso a paso, permiten controlar corriente y micropasos. Son compatibles con motores NEMA y reciben señales STEP/DIR desde el microcontrolador. Referencia \hyperref[ref2]{[2]}.

\subsection{Motores paso a paso}
\begin{itemize}
  \item \textbf{17HS3430}: torque nominal 26 Ncm, corriente 1.2A/fase, 1.8° por paso. Ideal para eje X.
  \item \textbf{17HS8401}: torque 52 Ncm, corriente 1.8A/fase. Usados en ejes Y y Z.
\end{itemize}
Hojas de datos disponibles en referencias \hyperref[ref3]{[3]} y \hyperref[ref4]{[4]}.

\subsection{RAMPS 1.4}
Placa de expansión diseñada para impresoras 3D, permite conectar drivers A4988, motores, finales de carrera y fuentes de alimentación. Es compatible eléctricamente con el Arduino Mega, pero se adapta a la Blue Pill mediante cableado personalizado. Referencia \hyperref[ref5]{[5]}.

\subsection{Fines de carrera mecánicos}
Interruptores tipo “normalmente cerrados” (NC) conectados a los pines PA12, PA15 y PA11 del microcontrolador. Se usan para detectar los límites de cada eje y realizar homing.

\section{Anexos}

\begin{itemize}
  \item \textbf{Mapa de pines Blue Pill STM32F103C8T6}: Ver referencia \hyperref[ref6]{[6]}
  \item \textbf{Esquema RAMPS 1.4}: Ver referencia \hyperref[ref7]{[7]}
\end{itemize}

\section{Referencias}

\begin{enumerate}
  \item \label{ref1} STMicroelectronics. STM32F103C8 Datasheet. \\
  \url{https://www.st.com/en/microcontrollers-microprocessors/stm32f103c8.html}
  
  \item \label{ref2} Pololu Corporation. A4988 Stepper Motor Driver Carrier. \\
  \url{https://www.pololu.com/product/1182}
  
  \item \label{ref3} OMC StepperOnline. Motor NEMA 17HS3430 Datasheet. \\
  \url{https://www.alldatasheet.es/datasheet-pdf/download/1137270/MOTIONKING/17HS3430.html}
  
  \item \label{ref4} OMC StepperOnline. Motor NEMA 17HS8401 Datasheet. \\
  \url{https://www.alldatasheet.es/datasheet-pdf/download/1137270/MOTIONKING/17HS3430.html}
  
  \item \label{ref5} RepRap Community. RAMPS 1.4 Documentation. \\
  \url{https://reprap.org/wiki/RAMPS_1.4}
  
  \item \label{ref6} Arduino STM32 Community. Blue Pill Pinout Reference. \\
  \url{https://github.com/rogerclarkmelbourne/Arduino_STM32/wiki/Blue-Pill-Pinout}
  
  \item \label{ref7} RepRap Community. RAMPS 1.4 Schematic. \\
  \url{https://reprap.org/mediawiki/images/c/c4/RAMPS1.4schematic.png}
\end{enumerate}

\end{document}
