\documentclass[12pt]{article}
\usepackage[spanish]{babel}
\usepackage[utf8]{inputenc}
\usepackage{graphicx}
\usepackage{amsmath}
\usepackage{xcolor}
\usepackage{tikz}
\usetikzlibrary{arrows.meta, positioning, shapes}
\usepackage{url}
\usepackage{hyperref}
\usepackage{amssymb} % For \checkmark symbol
\hypersetup{
    colorlinks=true,
    linkcolor=black,
    citecolor=black,
    urlcolor=black,
    pdfborder={0 0 0}
}
\usepackage{geometry}
\usepackage{fancyhdr}
\usepackage{lastpage}
\usepackage{listings}
\usepackage{float}
\usepackage{longtable}
\usepackage{multirow}
\geometry{a4paper, margin=1.3cm}

\title{Anteproyecto: Impresora 3D Modificada para Grabado Láser CNC de PCB}
\author{Nombre del estudiante}
\date{\today}

% Configuración de encabezados y pies de página
\pagestyle{fancy}
\fancyhf{} % Limpia encabezados y pies de página

% Encabezado
\fancyhead[L]{\scriptsize UNCuyo - Ing. Mecatrónica\\Mendoza - Argentina}
\fancyhead[C]{\scriptsize\textbf{\textbf{Microcontroladores y Electrónica de Potencia (Año 2024)}\\
\textbf{PROYECTO GLOBAL INTEGRADOR}}}
\fancyhead[R]{\scriptsize Peña Lautaro, Peralta Bruno\\19/07/2025}

% Pie de página
\fancyfoot[C]{Pág. \thepage\ de \pageref{LastPage}}

\setlength{\headheight}{16.9528pt}  % importante si usás fancyhdr con varios renglones
\setlength{\headsep}{0.5cm}    % separación entre cabecera y texto

% \fancyhead[L]{Automática y Máquinas Eléctricas} % Encabezado izquierdo
% \fancyhead[C]{2025} % Encabezado central
% \fancyhead[R]{Peña Lautaro, Peralta Bruno} % Encabezado derecho
% \fancyfoot[C]{\thepage} % Número de página centrado en el pie de página

% Configuración de código
\lstset{
    backgroundcolor=\color{gray!10},
    basicstyle=\ttfamily\footnotesize,
    breaklines=true,
    captionpos=b,
    commentstyle=\color{green!40!black},
    escapeinside={\%*}{*},
    frame=single,
    keywordstyle=\color{blue},
    numbers=left,
    numbersep=5pt,
    numberstyle=\tiny\color{gray},
    rulecolor=\color{black},
    showspaces=false,
    showstringspaces=false,
    stringstyle=\color{orange},
    tabsize=4
}

\begin{document}

% Portada
\begin{titlepage}
    \centering

    % Logo
    \includegraphics[width=0.8\textwidth]{img/fing.jpeg}\par\vspace{1cm}
    \vspace{2cm}
    % Título
    {\Huge \textbf{Proyecto Global Integrador:}\\[0.5cm]
    \textbf{CNC Laser de 3 Ejes}\vspace{2cm}}
    % \textbf{Motor Sincrónico de Imanes Permanentes}\par}\vspace{2cm}
    
    % Autores
    {\Large Peña Lautaro - 13099\\
    Peralta Bruno - 13220\par}\vspace{0.5cm}
    
    % Profesor titular
    {Año 2024\par}
\end{titlepage}


\newpage

% Índice
\tableofcontents
\newpage


% Introducción
\section{Introducción}

El presente proyecto consiste en la modificación de una impresora 3D antigua con el fin de reutilizar su estructura y componentes para desarrollar una máquina CNC de grabado láser destinada a la creación rápida de placas de circuito impreso (PCB). Esta herramienta permitirá fabricar prototipos de forma ágil para proyectos de electrónica y robótica.

La motivación principal radica en el desafío de aprovechar un equipo en desuso y otorgarle una nueva funcionalidad, aportando una solución económica y eficiente para la etapa de prototipado de circuitos. El sistema combina conocimientos de control de motores, programación de microcontroladores, comunicación serial y ejecución de comandos G-code.

\section{Esquema Tecnológico}

\begin{center}
\begin{tikzpicture}[node distance=3cm, auto]
  % Bloques principales
  \node[draw, rectangle, rounded corners, fill=blue!10] (pc) {PC con G-code};
  \node[draw, rectangle, rounded corners, fill=green!10, below of=pc, yshift=-0.5cm] (bluepill) {Blue Pill STM32F103C8T6};
  \node[draw, rectangle, rounded corners, fill=yellow!20, below of=bluepill, yshift=-0.5cm] (ramp) {RAMPS 1.4};
  \node[draw, rectangle, rounded corners, fill=orange!10, below of=ramp, xshift=-5.5cm, yshift=-0.5cm] (driverX) {Driver A4988 - Eje X};
  \node[draw, rectangle, rounded corners, fill=orange!10, below of=ramp, xshift=0cm, yshift=-0.5cm] (driverY) {Driver A4988 - Eje Y};
  \node[draw, rectangle, rounded corners, fill=orange!10, below of=ramp, xshift=5.5cm, yshift=-0.5cm] (driverZ) {Driver A4988 - Eje Z};
  \node[draw, rectangle, rounded corners, fill=red!10, below of=driverX, yshift=-0.5cm] (motorX) {M.PAP NEMA 17HS3430};
  \node[draw, rectangle, rounded corners, fill=red!10, below of=driverY, yshift=-0.5cm] (motorY) {M.PAP NEMA 17HS8401};
  \node[draw, rectangle, rounded corners, fill=red!10, below of=driverZ, yshift=-0.5cm] (motorZ) {M.PAP NEMA 17HS8401};
  \node[draw, rectangle, rounded corners, fill=gray!20, right of=pc, xshift=5cm] (serial) {Monitor Serial};
  \node[draw, rectangle, rounded corners, fill=cyan!20, right of=ramp, xshift=6cm] (endstops) {Fines de carrera};

  % Conexiones
  \draw[-{Stealth}] (pc) -- (bluepill) node[midway, left] {USB Serial};
  \draw[-{Stealth}] (bluepill) -- (ramp) node[midway, left] {Interfaz};
  \draw[-{Stealth}] (ramp) -- (driverX);
  \draw[-{Stealth}] (ramp) -- (driverY);
  \draw[-{Stealth}] (ramp) -- (driverZ);
  \draw[-{Stealth}] (driverX) -- (motorX);
  \draw[-{Stealth}] (driverY) -- (motorY);
  \draw[-{Stealth}] (driverZ) -- (motorZ);
  \draw[-{Stealth}, dashed] (ramp) -- (endstops) node[midway, above] {Sensores};
  \draw[-{Stealth}, dashed] (pc) -- (serial);
\end{tikzpicture}
\end{center}

\section{Detalle de Módulos}

\subsection{Microcontrolador: Blue Pill STM32F103C8T6}
Microcontrolador de 32 bits basado en ARM Cortex-M3, con 72 MHz, 64KB Flash y 20KB RAM. Opera a 3.3V, pero varios pines son tolerantes a 5V. Se programa mediante STM32CubeIDE. Más información en referencia \hyperref[ref1]{[1]}.

\subsection{Drivers A4988}
Módulos controladores de motores paso a paso, permiten controlar corriente y micropasos. Son compatibles con motores NEMA y reciben señales STEP/DIR desde el microcontrolador. Referencia \hyperref[ref2]{[2]}.

\subsection{Motores paso a paso}
\begin{itemize}
  \item \textbf{17HS3430}: torque nominal 26 Ncm, corriente 1.2A/fase, 1.8° por paso. Ideal para eje X.
  \item \textbf{17HS8401}: torque 52 Ncm, corriente 1.8A/fase. Usados en ejes Y y Z.
\end{itemize}
Hojas de datos disponibles en referencias \hyperref[ref3]{[3]} y \hyperref[ref4]{[4]}.

\subsection{RAMPS 1.4}
Placa de expansión diseñada para impresoras 3D, permite conectar drivers A4988, motores, finales de carrera y fuentes de alimentación. Es compatible eléctricamente con el Arduino Mega, pero se adapta a la Blue Pill mediante cableado personalizado. Referencia \hyperref[ref5]{[5]}.

\subsection{Fines de carrera mecánicos}
Interruptores tipo “normalmente cerrados” (NC) conectados a los pines PA12, PA15 y PA11 del microcontrolador. Se usan para detectar los límites de cada eje y realizar homing.


% Funcionamiento general
\section{Funcionamiento general}
\subsection*{Interacción entre módulos y funciones}

\begin{itemize}
    \item \textbf{PC con G-code:} Envía comandos G-code a través de USB Serial. Puede utilizar un monitor serial para control manual y monitoreo.
    \item \textbf{Blue Pill STM32F103C8T6:} Actúa como controlador principal, interpreta los comandos G-code, genera señales de paso y dirección para los motores, y procesa la información de los sensores de fin de carrera.
    \item \textbf{RAMPS 1.4:} Placa de interfaz que conecta el microcontrolador con los drivers de motores y los sensores de límite.
    \item \textbf{Drivers A4988:} Uno por cada eje (X, Y, Z), reciben señales STEP/DIR y controlan la corriente y microstepping de los motores.
    \item \textbf{Motores paso a paso NEMA:} Ejecutan el movimiento mecánico.
    \item \textbf{Sensores de fin de carrera:} Detectan la posición de referencia (\textit{home}) y evitan movimientos fuera del rango mecánico.
\end{itemize}

\subsection*{Diagrama de flujo}

% Diagrama de flujo del funcionamiento general
\begin{center}
\begin{tikzpicture}[
    scale=0.7,
    transform shape,
    node distance=1.5cm,
    every node/.style={font=\scriptsize},
    process/.style={rectangle, draw, fill=blue!10, rounded corners, minimum width=2.2cm, minimum height=0.7cm, align=center},
    decision/.style={draw, shape=diamond, fill=yellow!20, inner sep=0.15cm, align=center},
    startstop/.style={rectangle, draw, fill=green!20, rounded corners, minimum width=2.2cm, minimum height=0.7cm, align=center},
    arrow/.style={-Stealth, thick}
]

% Nodos
\node[startstop] (start) {Inicio sistema};
\node[process, below of=start] (init) {Inicializar módulos};
\node[process, below of=init] (recv) {Recibir G-code desde PC};
\node[process, below of=recv] (interp) {Interpretar comando};
\node[decision, below of=interp, yshift=-2cm] (decision1) {¿Es comando de movimiento?};
\node[process, right=4.5cm of decision1] (action) {Ejecutar acción\\(ej: encender láser)};
\node[process, below=1cm of decision1] (calc) {Calcular trayectoria\\y pasos de motores};
\node[process, below of=calc] (send) {Enviar señales a drivers};
\node[process, below of=send] (read) {Leer sensores de límites};
\node[decision, below of=read, yshift=-2cm] (decision2) {¿Fin de carrera activado?};
\node[process, left=4.5cm of decision2] (stopm) {Detener movimiento};
\node[process, below=1cm of decision2] (wait) {Esperar nuevo comando};
\node[startstop, below=1cm of stopm] (error) {Error};

% Flechas
\draw[arrow] (start) -- (init);
\draw[arrow] (init) -- (recv);
\draw[arrow] (recv) -- (interp);
\draw[arrow] (interp) -- (decision1);
\draw[arrow] (decision1.east) -- node[above]{No} (action.west);
\draw[arrow] (decision1.south) -- node[left]{Sí} (calc.north);
\draw[arrow] (action.south) |- (wait.east);
\draw[arrow] (calc) -- (send);
\draw[arrow] (send) -- (read);
\draw[arrow] (read) -- (decision2);
\draw[arrow] (decision2.west) -- node[above]{Sí} (stopm.east);
\draw[arrow] (decision2.south) -- node[left]{No} (wait.north);
\draw[arrow] (stopm.south) -- (error.north);
\draw[arrow] (wait.west) -- ++(-9,0) |- (interp.west);




\end{tikzpicture}
\end{center}

% \end{tikzpicture}

% Programación
\section{Programación}

\subsection{Configuración de Periféricos del Controlador STM32F103C8T6}

\subsubsection{Interfaces de Comunicación}

\textbf{USB CDC (Communication Device Class):} el sistema utiliza comunicación USB para interactuar con el software de control externo (software host), implementando un puerto serial virtual.

\begin{itemize}
    \item \textbf{Pines:} PA11 (USB\_DM), PA12 (USB\_DP)
    \item \textbf{Funcionalidad:} Comunicación serie virtual via USB
    \item \textbf{Buffer:} RX=1024 bytes, TX=1024 bytes
    \item \textbf{Sistema de Cola:} Implementado para transmisión confiable
    \item \textbf{Interrupción:} \texttt{USB\_LP\_CAN1\_RX0\_IRQn} habilitada
\end{itemize}

\subsubsection{Control de Motores Paso a Paso}

\textbf{Configuración GPIO para Motores:} el sistema controla 3 motores paso a paso (ejes X, Y, Z) mediante pines GPIO configurados como salidas push-pull:

\begin{table}[h]
\centering
\begin{tabular}{|l|c|c|c|c|}
\hline
\textbf{Eje} & \textbf{STEP} & \textbf{DIR} & \textbf{EN} & \textbf{Endstop} \\
\hline
X & PB6 & PB7 & PA8 & PB12 \\
Y & PB9 & PB3 & PB4 & PB13 \\
Z & PA8 & PA9 & PA10 & PB14 \\
\hline
\end{tabular}
\caption{Configuración de pines para control de motores}
\end{table}

\textbf{Características de Timing:}
\begin{itemize}
    \item \textbf{Delay base entre pasos}: 800 µs (configurable)
    \item \textbf{Pulso de STEP}: 2 µs de duración mínima
    \item \textbf{Velocidad máxima}: Limitada por \texttt{calculateStepDelay()} basado en feed rate
    \item \textbf{Timing preciso}: Utilización del contador DWT para delays de microsegundos
\end{itemize}

\subsubsection{Sistema de Interrupciones Externas}

\textbf{Finales de Carrera:} configuración de interrupciones para detección de límites físicos:

\begin{itemize}
    \item \textbf{Pines}: PB12, PB13, PB14
    \item \textbf{Modo}: Interrupción por flanco ascendente
    \item \textbf{Pull-up}: Resistencias pull-up internas habilitadas
    \item \textbf{Prioridad}: Prioridad 0 (máxima) para respuesta inmediata
    \item \textbf{Debounce}: Implementado por software (50ms)
\end{itemize}

\subsubsection{Indicadores LED}

\textbf{LEDs de Estado}
\begin{itemize}
    \item \textbf{LED\_CHECK} (PB1): Indicador de sistema activo
    \item \textbf{LED\_ERROR} (PB0): Indicador de errores del sistema
    \item \textbf{Configuración}: Salidas push-pull, activo en alto
\end{itemize}

\subsubsection{Configuración del Sistema de Reloj}

\begin{itemize}
    \item \textbf{Oscilador Externo (HSE):} 8 MHz con PLL x6 = 48 MHz de frecuencia del sistema
    \item \textbf{AHB Clock:} 48 MHz
    \item \textbf{APB1 Clock:} 24 MHz (dividido por 2)
    \item \textbf{APB2 Clock:} 48 MHz
    \item \textbf{USB Clock:} Derivado del PLL para operación a 48 MHz
\end{itemize}


\subsection{Módulos Externos}

\subsubsection{Drivers de Motores Paso a Paso}
\begin{itemize}
    \item \textbf{Tipo:} Compatibles con señales STEP/DIR/ENABLE
    \item \textbf{Control:} Lógica activa alta para STEP y DIR
    \item \textbf{Enable:} Lógica activa baja (LOW = habilitado)
    \item \textbf{Timing:} Delay de 800 \textmu{}s entre pulsos configurables
\end{itemize}

\subsubsection{Finales de Carrera}
\begin{itemize}
    \item \textbf{Tipo:} Switches normalmente abiertos
    \item \textbf{Conexión:} Con pull-up interno del STM32
    \item \textbf{Detección:} Nivel bajo indica activación
    \item \textbf{Funcionalidad:} Homing y detección de límites
\end{itemize}


\subsection{Descripción de Funciones Codificadas}

\subsubsection{Función main()}

\paragraph{Estructura Principal}
La función \texttt{main()} implementa la inicialización del sistema y el bucle principal:

\begin{itemize}
    \item \textbf{Inicialización HAL}: \texttt{HAL\_Init()}
    \item \textbf{Configuración de reloj}: \texttt{SystemClock\_Config()}
    \item \textbf{Inicialización de periféricos}: GPIO, UART, USB
    \item \textbf{Setup personalizado}: Configuración específica del CNC
    \item \textbf{Bucle infinito}: Procesamiento continuo de comandos y estados
\end{itemize}

\subsubsection{Algoritmos de Control de Movimiento}

\textbf{Algoritmo de Bresenham Modificado:} implementado en \texttt{motion.c} para movimientos lineales coordenados.

\begin{enumerate}
    \item Cálculo de diferencias absolutas entre coordenadas
    \item Determinación del eje dominante (mayor distancia)
    \item Interpolación lineal para mantener proporcionalidad
    \item Control de timing basado en feed rate especificado
\end{enumerate}

\textbf{Generación de Arcos:} Algoritmo para movimientos circulares (G2/G3).

\begin{itemize}
    \item \textbf{Segmentación}: División del arco en 50 segmentos lineales
    \item \textbf{Cálculo trigonométrico}: Uso de seno y coseno para interpolación
    \item \textbf{Precisión}: Balance entre suavidad y carga computacional
\end{itemize}

\subsubsection{Parser G-code}

\textbf{Análisis Léxico y Sintáctico:} implementado en \texttt{gcode\_parser.c} con las siguientes características.

\begin{itemize}
    \item \textbf{Parsing modular}: Separación de análisis y ejecución
    \item \textbf{Validación de grupos modales}: Verificación de compatibilidad de comandos
    \item \textbf{Gestión de estado}: Mantenimiento de estado modal persistente
    \item \textbf{Verificación de límites}: Control de límites software en tiempo real
\end{itemize}

\textbf{Comandos Soportados:}
\begin{itemize}
    \item \textbf{G0}: Movimiento rápido (rapid positioning)
    \item \textbf{G1}: Movimiento lineal con feed rate
    \item \textbf{G2/G3}: Movimientos circulares horario/antihorario
    \item \textbf{G28}: Homing de todos los ejes
    \item \textbf{M114}: Reporte de posición actual
    \item \textbf{M119}: Estado de finales de carrera
    \item \textbf{M503}: Mostrar configuración del sistema
\end{itemize}

\subsubsection{Sistema de Interrupciones}

\textbf{Manejo de Finales de Carrera:} función \texttt{HAL\_GPIO\_EXTI\_Callback()}.

\begin{enumerate}
    \item Identificación del pin que generó la interrupción
    \item Implementación de debounce por software (50ms)
    \item Parada inmediata del movimiento en curso
    \item Generación de mensaje de error específico por eje
    \item Activación del LED de error
\end{enumerate}

\textbf{Interrupción del Sistema (SysTick):}
\begin{itemize}
    \item \textbf{Frecuencia}: 1ms (1kHz)
    \item \textbf{Funcionalidad}: Base de tiempo para \texttt{HAL\_GetTick()}
    \item \textbf{Uso}: Timeouts, delays no bloqueantes
\end{itemize}

\subsubsection{Algoritmo de Homing}

\textbf{Secuencia de Homing Automático:} implementado en \texttt{performHoming()}.

\begin{enumerate}
    \item \textbf{Fase 1}: Movimiento rápido hacia finales de carrera
    \item \textbf{Fase 2}: Retroceso de seguridad (2-4mm según el eje)
    \item \textbf{Fase 3}: Aproximación lenta y precisa al final de carrera
    \item \textbf{Fase 4}: Establecimiento de coordenadas de origen (0,0,0)
\end{enumerate}

\textbf{Verificaciones de Seguridad:}
\begin{itemize}
    \item Verificación de liberación de endstops tras retroceso
    \item Confirmación de activación tras aproximación final
    \item Generación de mensajes de error específicos
    \item Deshabilitación temporal de interrupciones durante homing
\end{itemize}

\subsubsection{Sistema de Gestión de Programas}

\textbf{Almacenamiento de Programas G-code:}
\begin{itemize}
    \item \textbf{Capacidad}: 100 líneas máximo (\texttt{MAX\_GCODE\_LINES})
    \item \textbf{Longitud por línea}: Configurable (\texttt{MAX\_LINE\_LENGTH})
    \item \textbf{Comandos de control}: \texttt{PROGRAM\_START}, \texttt{PROGRAM\_STOP}, \texttt{PROGRAM\_RUN}
    \item \textbf{Validación}: Verificación de sintaxis y límites durante almacenamiento
\end{itemize}

\textbf{Ejecución Progresiva:} función \texttt{processProgram()}.

\begin{itemize}
    \item Ejecución línea por línea sin bloqueo del sistema principal
    \item Control de flujo con comandos de pausa y reanudación
    \item Reporte de progreso y estado de ejecución
    \item Manejo de errores con detención automática
\end{itemize}

\subsubsection{Sistema de Comunicaciones}

\textbf{Cola de Transmisión USB:} implementación en \texttt{usbd\_cdc\_if.c}.

\begin{itemize}
    \item \textbf{Buffering}: Sistema de colas para evitar pérdida de datos
    \item \textbf{Procesamiento}: Función \texttt{CDC\_TxQueue\_Process()} no bloqueante
    \item \textbf{Gestión de flujo}: Control automático de flujo de datos
    \item \textbf{Robustez}: Manejo de desconexiones y reconexiones USB
\end{itemize}

\textbf{Protocolo de Comunicación:}
\begin{itemize}
    \item \textbf{Formato}: Compatible con protocolo GRBL estándar
    \item \textbf{Respuestas}: 'ok' para comandos exitosos, códigos de error específicos
    \item \textbf{Reportes}: Estados de máquina, posiciones, configuración
    \item \textbf{Comandos especiales}: Sistema de help y diagnóstico
\end{itemize}



% Etapas de montaje y ensayos realizados
\section{Etapas de montaje y ensayos realizados}
Describir las pruebas parciales o conjuntas realizadas.

% Resultados
\section{Resultados y especificaciones finales}

\subsection{Grado de Cumplimiento de Metas y Objetivos}

\subsubsection{Objetivos Principales Alcanzados}

\textbf{Control de Movimiento Coordenado:} el sistema implementa control completo de 3 ejes (X, Y, Z) con las siguientes capacidades.
\begin{itemize}
    \item Movimiento coordinado mediante algoritmo de Bresenham modificado
    \item Interpolación lineal para movimientos G1 con feed rate variable
    \item Movimientos rápidos G0 con velocidad optimizada
    \item Movimientos circulares G2/G3 con segmentación automática
    \item Sistema de coordenadas absoluto con mantenimiento de posición
\end{itemize}

\textbf{Procesamiento de Comandos G-code:} se implementó un parser G-code profesional compatible con estándar GRBL que soporta.
\begin{itemize}
    \item \textbf{Comandos de movimiento}: G0, G1, G2, G3
    \item \textbf{Comandos de sistema}: G28 (homing)
    \item \textbf{Gestión de programas}: Almacenamiento y ejecución de hasta 100 líneas
    \item \textbf{Validación}: Verificación sintáctica y de límites en tiempo real
\end{itemize}

\textbf{Comunicación USB:} se implementó un sistema de comunicación robusto.
\begin{itemize}
    \item Puerto serial virtual USB CDC
    \item Sistema de colas de transmisión para prevenir pérdida de datos
    \item Protocolo compatible con software CAM estándar
    \item Manejo de reconexiones automático
    \item Buffer bidireccional de 1024 bytes
\end{itemize}

\textbf{Sistema de Seguridad:} se implementó un sistema de seguridad completo.
\begin{itemize}
    \item Finales de carrera en los 3 ejes con interrupción inmediata
    \item Límites de software configurables por eje
    \item Secuencia de homing automático de alta precisión
    \item Debounce por software (50ms) para finales de carrera
    \item Validación proactiva de límites durante carga de programas
\end{itemize}

\subsubsection{Objetivos Secundarios}

\textbf{Indicadores Visuales:}
\begin{itemize}
    \item LED de estado del sistema (LED\_CHECK)
    \item LED de error para fallas (LED\_ERROR)
    \item Secuencia de inicialización visual (3 parpadeos)
\end{itemize}

\textbf{Sistema de Debug:}
\begin{itemize}
    \item Mensajes de diagnóstico configurables
    \item Comando HELP integrado
    \item Reporte de estado de sistema (QUEUE\_STATUS)
\end{itemize}

\subsection{Especificaciones Técnicas Alcanzadas}

\subsubsection{Resolución de Movimiento}

\begin{table}[h]
\centering
\begin{tabular}{|l|c|c|}
\hline
\textbf{Eje} & \textbf{Pasos/mm} & \textbf{Resolución} \\
\hline
X & 79 & 0.0127 mm \\
Y & 79 & 0.0127 mm \\
Z & 3930 & 0.000254 mm \\
\hline
\end{tabular}
\caption{Resolución por eje del sistema}
\end{table}

\subsubsection{Velocidad y Rendimiento}

\paragraph{Velocidades de Operación}
\begin{itemize}
    \item \textbf{Velocidad por defecto (G1)}: Eje X, Y 15 mm/s; Eje Z 0.318 mm/s
    \item \textbf{Velocidad rápida y máxima(G0)}: Eje X, Y 63 mm/s; Eje Z 1.27 mm/s (configurable)
\end{itemize}

\paragraph{Tiempos de Respuesta}
\begin{itemize}
    \item \textbf{Respuesta a comando}: <10 ms
    \item \textbf{Detección de endstop}: <1 ms (interrupción hardware)
    \item \textbf{Tiempo de homing completo}: 15-30 segundos (según dimensiones)
    \item \textbf{Frecuencia de procesamiento}: 50 Hz (loop principal)
\end{itemize}

\subsubsection{Rangos de Trabajo}

\textbf{Área de Trabajo Configurada:}
\begin{table}[h]
\centering
\begin{tabular}{|l|c|c|}
\hline
\textbf{Eje} & \textbf{Mínimo} & \textbf{Máximo} \\
\hline
X & 0.0 mm & 160.0 mm \\
Y & 0.0 mm & 160.0 mm \\
Z & 0.0 mm & 160.0 mm \\
\hline
\textbf{Volumen total} & \multicolumn{2}{c|}{4.096 litros (160³ mm³)} \\
\hline
\end{tabular}
\caption{Límites de trabajo del sistema CNC}
\end{table}

\newpage
\textbf{Capacidades de Almacenamiento:}
\begin{itemize}
    \item \textbf{Programa G-code}: Hasta 100 líneas simultáneas
    \item \textbf{Longitud por línea}: 100 caracteres máximo
    \item \textbf{Buffer de comunicación}: 1024 bytes bidireccional
    \item \textbf{Cola de transmisión}: Implementada para prevenir pérdida
\end{itemize}

\subsubsection{Manejo de Errores y Estabilidad}
\begin{itemize}
    \item \textbf{Recuperación automática}: Sistema de watchdog implícito
    \item \textbf{Protección contra sobrecarga}: Límites de posición y velocidad
    \item \textbf{Manejo de reconexión USB}: Automático sin pérdida de estado

\end{itemize}


\subsection{Limitaciones Identificadas}

\textbf{Velocidad de Procesamiento:}
\begin{itemize}
    \item Segmentación de arcos fija (50 segmentos) puede limitar suavidad
    \item Sin aceleración/desaceleración progresiva (trapezoidal)
    \item Tiempo mínimo entre pasos de 200µs limita velocidad máxima teórica
\end{itemize}

\textbf{Funcionalidades Pendientes:}
\begin{itemize}
    \item Control PWM para láser (preparado pero no implementado)
    \item Compensación de backlash mecánico
    \item Interpolación cúbica para figuras complejas
    \item Perfiles de velocidad trapezoidal
\end{itemize}

\subsection{Comparación con Objetivos Iniciales}

\begin{table}[h]
\centering
\begin{tabular}{|l|c|c|}
\hline
\textbf{Objetivo} & \textbf{Meta} & \textbf{Alcanzado} \\
Control 3 ejes & Sí & \checkmark~100\% \\
Comunicación USB & Sí & \checkmark~100\% \\
Parser G-code básico & G0, G1 & \checkmark~G0,G1,G2,G3 (120\%) \\
Finales de carrera & 3 ejes & \checkmark~95\% \\
Precisión & 0.1 mm & \checkmark~0.02 mm \\
Velocidad & 30 mm/s & \checkmark~63 mm/s \\
Sistema de seguridad & Básico & \checkmark~Avanzado \\
Área de trabajo & 100x100 mm & \checkmark~160x160x160 mm \\
\hline
\end{tabular}
\caption{Comparación objetivos vs. resultados}
\end{table}

\newpage
% Conclusiones
\section{Conclusiones. Ensayo de ingeniería de producto.}
Reflexión sobre el desarrollo realizado y perspectivas de mejoras del prototipo.  
Ejercicio de selección de componentes para su paso a un producto comercial o sistema de aplicación industrial o del ámbito que corresponda.

\subsection{Cumplimiento de Objetivos}

El proyecto cumplió satisfactoriamente con todos los objetivos planteados:

\begin{itemize}
    \item Se desarrolló un sistema CNC funcional basado en STM32F103C8T6
    \item El firmware adaptado de GRBL permite interpretación completa de G-code
    \item La precisión alcanzada es adecuada para las aplicaciones objetivo
\end{itemize}

\subsection{Aprendizajes Obtenidos}

Durante el desarrollo del proyecto se adquirieron conocimientos en:
\begin{itemize}
    \item Programación de microcontroladores ARM Cortex-M3
    \item Control de motores paso a paso y sistemas de movimiento
    \item Interpretación y procesamiento de código G-code
    \item Integración de hardware y software en sistemas embebidos
    \item Técnicas de debugging y optimización en sistemas de tiempo real
\end{itemize}

\subsection{Trabajo Futuro}

Las siguientes mejoras podrían implementarse en versiones futuras:

\subsubsection{Hardware}
\begin{itemize}
    \item Implementación de encoders para retroalimentación de posición
    \item Adición de sensores de temperatura y corriente
    \item Upgrade a drivers de motor más avanzados (TMC2209, TMC5160)
    \item Sistema de refrigeración para cortes prolongados
\end{itemize}

\subsubsection{Software}
\begin{itemize}
    \item Implementación de compensación de backlash
    \item Algoritmos de aceleración adaptativa
    \item Interface web para control remoto
    \item Sistema de detección de colisiones
\end{itemize}

\subsubsection{Mecánica}
\begin{itemize}
    \item Estructura más rígida 
    \item Aumento del área de trabajo a 400x400x100mm
    \item Sistema de sujeción de piezas
\end{itemize}

% Referencias
\section{Referencias}
Bibliografía, hojas de datos, guías, enlaces a sitios o documentos de internet.

\begin{enumerate}
  \item \label{ref1} STMicroelectronics. STM32F103C8 Datasheet. \\
  \url{https://www.st.com/en/microcontrollers-microprocessors/stm32f103c8.html}
  
  \item \label{ref2} Pololu Corporation. A4988 Stepper Motor Driver Carrier. \\
  \url{https://www.pololu.com/product/1182}
  
  \item \label{ref3} OMC StepperOnline. Motor NEMA 17HS3430 Datasheet. \\
  \url{https://www.alldatasheet.es/datasheet-pdf/download/1137270/MOTIONKING/17HS3430.html}
  
  \item \label{ref4} OMC StepperOnline. Motor NEMA 17HS8401 Datasheet. \\
  \url{https://www.alldatasheet.es/datasheet-pdf/download/1137270/MOTIONKING/17HS3430.html}
  
  \item \label{ref5} RepRap Community. RAMPS 1.4 Documentation. \\
  \url{https://reprap.org/wiki/RAMPS_1.4}
  
  \item \label{ref6} Arduino STM32 Community. Blue Pill Pinout Reference. \\
  \url{https://github.com/rogerclarkmelbourne/Arduino_STM32/wiki/Blue-Pill-Pinout}
  
  \item \label{ref7} RepRap Community. RAMPS 1.4 Schematic. \\
  \url{https://reprap.org/mediawiki/images/c/c4/RAMPS1.4schematic.png}
\end{enumerate}


% Anexos
\section{Anexos}
Aquellos detalles no disponibles en las referencias de acceso público.

\begin{itemize}
  \item \textbf{Mapa de pines Blue Pill STM32F103C8T6}: Ver referencia \hyperref[ref6]{[6]}
  \item \textbf{Esquema RAMPS 1.4}: Ver referencia \hyperref[ref7]{[7]}
\end{itemize}

\end{document}
