\documentclass[12pt]{article}
\usepackage[spanish]{babel}
\usepackage[utf8]{inputenc}
\usepackage{graphicx}
\usepackage{amsmath}
\usepackage{xcolor}
\usepackage{tikz}
\usetikzlibrary{arrows.meta, positioning, shapes}
\usepackage{url}
\usepackage{hyperref}
\hypersetup{
    colorlinks=true,
    linkcolor=black,
    citecolor=black,
    urlcolor=black,
    pdfborder={0 0 0}
}
\usepackage{geometry}
\usepackage{fancyhdr}
\usepackage{lastpage}
\usepackage{listings}
\usepackage{float}
\usepackage{longtable}
\geometry{a4paper, margin=1.3cm}

\title{Anteproyecto: Impresora 3D Modificada para Grabado Láser CNC de PCB}
\author{Nombre del estudiante}
\date{\today}

% Configuración de encabezados y pies de página
\pagestyle{fancy}
\fancyhf{} % Limpia encabezados y pies de página

% Encabezado
\fancyhead[L]{\scriptsize UNCuyo - Ing. Mecatrónica\\Mendoza - Argentina}
\fancyhead[C]{\scriptsize\textbf{\textbf{Microcontroladores y Electrónica de Potencia (Año 2024)}\\
\textbf{PROYECTO GLOBAL INTEGRADOR}}}
\fancyhead[R]{\scriptsize Peña Lautaro, Peralta Bruno\\19/07/2025}

% Pie de página
\fancyfoot[C]{Pág. \thepage\ de \pageref{LastPage}}

\setlength{\headheight}{16.9528pt}  % importante si usás fancyhdr con varios renglones
\setlength{\headsep}{0.5cm}    % separación entre cabecera y texto

% \fancyhead[L]{Automática y Máquinas Eléctricas} % Encabezado izquierdo
% \fancyhead[C]{2025} % Encabezado central
% \fancyhead[R]{Peña Lautaro, Peralta Bruno} % Encabezado derecho
% \fancyfoot[C]{\thepage} % Número de página centrado en el pie de página

% Configuración de código
\lstset{
    backgroundcolor=\color{gray!10},
    basicstyle=\ttfamily\footnotesize,
    breaklines=true,
    captionpos=b,
    commentstyle=\color{green!40!black},
    escapeinside={\%*}{*},
    frame=single,
    keywordstyle=\color{blue},
    numbers=left,
    numbersep=5pt,
    numberstyle=\tiny\color{gray},
    rulecolor=\color{black},
    showspaces=false,
    showstringspaces=false,
    stringstyle=\color{orange},
    tabsize=4
}

\begin{document}

% Portada
\begin{titlepage}
    \centering

    % Logo
    \includegraphics[width=0.8\textwidth]{img/fing.jpeg}\par\vspace{1cm}
    \vspace{2cm}
    % Título
    {\Huge \textbf{Proyecto Global Integrador:}\\[0.5cm]
    \textbf{CNC Laser de 3 Ejes}\vspace{2cm}}
    % \textbf{Motor Sincrónico de Imanes Permanentes}\par}\vspace{2cm}
    
    % Autores
    {\Large Peña Lautaro - 13099\\
    Peralta Bruno - 13220\par}\vspace{0.5cm}
    
    % Profesor titular
    {Año 2024\par}
\end{titlepage}


\newpage

% Índice
\tableofcontents
\newpage


% Introducción
\section{Introducción}

El presente proyecto consiste en la modificación de una impresora 3D antigua con el fin de reutilizar su estructura y componentes para desarrollar una máquina CNC de grabado láser destinada a la creación rápida de placas de circuito impreso (PCB). Esta herramienta permitirá fabricar prototipos de forma ágil para proyectos de electrónica y robótica.

La motivación principal radica en el desafío de aprovechar un equipo en desuso y otorgarle una nueva funcionalidad, aportando una solución económica y eficiente para la etapa de prototipado de circuitos. El sistema combina conocimientos de control de motores, programación de microcontroladores, comunicación serial y ejecución de comandos G-code.

\section{Esquema Tecnológico}

\begin{center}
\begin{tikzpicture}[node distance=3cm, auto]
  % Bloques principales
  \node[draw, rectangle, rounded corners, fill=blue!10] (pc) {PC con G-code};
  \node[draw, rectangle, rounded corners, fill=green!10, below of=pc, yshift=-0.5cm] (bluepill) {Blue Pill STM32F103C8T6};
  \node[draw, rectangle, rounded corners, fill=yellow!20, below of=bluepill, yshift=-0.5cm] (ramp) {RAMPS 1.4};
  \node[draw, rectangle, rounded corners, fill=orange!10, below of=ramp, xshift=-5.5cm, yshift=-0.5cm] (driverX) {Driver A4988 - Eje X};
  \node[draw, rectangle, rounded corners, fill=orange!10, below of=ramp, xshift=0cm, yshift=-0.5cm] (driverY) {Driver A4988 - Eje Y};
  \node[draw, rectangle, rounded corners, fill=orange!10, below of=ramp, xshift=5.5cm, yshift=-0.5cm] (driverZ) {Driver A4988 - Eje Z};
  \node[draw, rectangle, rounded corners, fill=red!10, below of=driverX, yshift=-0.5cm] (motorX) {M.PAP NEMA 17HS3430};
  \node[draw, rectangle, rounded corners, fill=red!10, below of=driverY, yshift=-0.5cm] (motorY) {M.PAP NEMA 17HS8401};
  \node[draw, rectangle, rounded corners, fill=red!10, below of=driverZ, yshift=-0.5cm] (motorZ) {M.PAP NEMA 17HS8401};
  \node[draw, rectangle, rounded corners, fill=gray!20, right of=pc, xshift=5cm] (serial) {Monitor Serial};
  \node[draw, rectangle, rounded corners, fill=cyan!20, right of=ramp, xshift=6cm] (endstops) {Fines de carrera};

  % Conexiones
  \draw[-{Stealth}] (pc) -- (bluepill) node[midway, left] {USB Serial};
  \draw[-{Stealth}] (bluepill) -- (ramp) node[midway, left] {Interfaz};
  \draw[-{Stealth}] (ramp) -- (driverX);
  \draw[-{Stealth}] (ramp) -- (driverY);
  \draw[-{Stealth}] (ramp) -- (driverZ);
  \draw[-{Stealth}] (driverX) -- (motorX);
  \draw[-{Stealth}] (driverY) -- (motorY);
  \draw[-{Stealth}] (driverZ) -- (motorZ);
  \draw[-{Stealth}, dashed] (ramp) -- (endstops) node[midway, above] {Sensores};
  \draw[-{Stealth}, dashed] (pc) -- (serial);
\end{tikzpicture}
\end{center}

\section{Detalle de Módulos}

\subsection{Microcontrolador: Blue Pill STM32F103C8T6}
Microcontrolador de 32 bits basado en ARM Cortex-M3, con 72 MHz, 64KB Flash y 20KB RAM. Opera a 3.3V, pero varios pines son tolerantes a 5V. Se programa mediante STM32CubeIDE. Más información en referencia \hyperref[ref1]{[1]}.

\subsection{Drivers A4988}
Módulos controladores de motores paso a paso, permiten controlar corriente y micropasos. Son compatibles con motores NEMA y reciben señales STEP/DIR desde el microcontrolador. Referencia \hyperref[ref2]{[2]}.

\subsection{Motores paso a paso}
\begin{itemize}
  \item \textbf{17HS3430}: torque nominal 26 Ncm, corriente 1.2A/fase, 1.8° por paso. Ideal para eje X.
  \item \textbf{17HS8401}: torque 52 Ncm, corriente 1.8A/fase. Usados en ejes Y y Z.
\end{itemize}
Hojas de datos disponibles en referencias \hyperref[ref3]{[3]} y \hyperref[ref4]{[4]}.

\subsection{RAMPS 1.4}
Placa de expansión diseñada para impresoras 3D, permite conectar drivers A4988, motores, finales de carrera y fuentes de alimentación. Es compatible eléctricamente con el Arduino Mega, pero se adapta a la Blue Pill mediante cableado personalizado. Referencia \hyperref[ref5]{[5]}.

\subsection{Fines de carrera mecánicos}
Interruptores tipo “normalmente cerrados” (NC) conectados a los pines PA12, PA15 y PA11 del microcontrolador. Se usan para detectar los límites de cada eje y realizar homing.


% Funcionamiento general
\section{Funcionamiento general}
\subsection*{Módulos y funciones}

\begin{itemize}
    \item \textbf{PC con G-code:} Envía comandos G-code a través de USB Serial. Puede utilizar un monitor serial para control manual y monitoreo.
    \item \textbf{Blue Pill STM32F103C8T6:} Actúa como controlador principal, interpreta los comandos G-code, genera señales de paso y dirección para los motores, y procesa la información de los sensores de fin de carrera.
    \item \textbf{RAMPS 1.4:} Placa de interfaz que conecta el microcontrolador con los drivers de motores y los sensores de límite.
    \item \textbf{Drivers A4988:} Uno por cada eje (X, Y, Z), reciben señales STEP/DIR y controlan la corriente y microstepping de los motores.
    \item \textbf{Motores paso a paso NEMA:} Ejecutan el movimiento mecánico.
    \item \textbf{Sensores de fin de carrera:} Detectan la posición de referencia (\textit{home}) y evitan movimientos fuera del rango mecánico.
\end{itemize}

% \section*{Diagrama de flujo}

% Diagrama de flujo del funcionamiento general
\begin{center}
\begin{tikzpicture}[
    scale=0.7,
    transform shape,
    node distance=1.5cm,
    every node/.style={font=\scriptsize},
    process/.style={rectangle, draw, fill=blue!10, rounded corners, minimum width=2.2cm, minimum height=0.7cm, align=center},
    decision/.style={draw, shape=diamond, fill=yellow!20, inner sep=0.15cm, align=center},
    startstop/.style={rectangle, draw, fill=green!20, rounded corners, minimum width=2.2cm, minimum height=0.7cm, align=center},
    arrow/.style={-Stealth, thick}
]

% Nodos
\node[startstop] (start) {Inicio sistema};
\node[process, below of=start] (init) {Inicializar módulos};
\node[process, below of=init] (recv) {Recibir G-code desde PC};
\node[process, below of=recv] (interp) {Interpretar comando};
\node[decision, below of=interp, yshift=-2cm] (decision1) {¿Es comando de movimiento?};
\node[process, right=4.5cm of decision1] (action) {Ejecutar acción\\(ej: encender láser)};
\node[process, below=1cm of decision1] (calc) {Calcular trayectoria\\y pasos de motores};
\node[process, below of=calc] (send) {Enviar señales a drivers};
\node[process, below of=send] (read) {Leer sensores de límites};
\node[decision, below of=read, yshift=-2cm] (decision2) {¿Fin de carrera activado?};
\node[process, left=4.5cm of decision2] (stopm) {Detener movimiento};
\node[process, below=1cm of decision2] (wait) {Esperar nuevo comando};
\node[startstop, below=1cm of stopm] (error) {Error};

% Flechas
\draw[arrow] (start) -- (init);
\draw[arrow] (init) -- (recv);
\draw[arrow] (recv) -- (interp);
\draw[arrow] (interp) -- (decision1);
\draw[arrow] (decision1.east) -- node[above]{No} (action.west);
\draw[arrow] (decision1.south) -- node[left]{Sí} (calc.north);
\draw[arrow] (action.south) |- (wait.east);
\draw[arrow] (calc) -- (send);
\draw[arrow] (send) -- (read);
\draw[arrow] (read) -- (decision2);
\draw[arrow] (decision2.west) -- node[above]{Sí} (stopm.east);
\draw[arrow] (decision2.south) -- node[left]{No} (wait.north);
\draw[arrow] (stopm.south) -- (error.north);
\draw[arrow] (wait.west) -- ++(-9,0) |- (interp.west);




\end{tikzpicture}
\end{center}

% \end{tikzpicture}

% Programación
\section{Programación}
Detalle de configuración de periféricos del controlador (interfaces de comunicación, A/D, PWM, etc.) y de módulos externos (sensores, drivers, conversores, memorias, etc.).  
Descripción de las funciones codificadas (\texttt{main()}, algoritmos, diagramas de estado, interrupciones, etc.).

% Etapas de montaje y ensayos realizados
\section{Etapas de montaje y ensayos realizados}
Describir las pruebas parciales o conjuntas realizadas.

% Resultados
\section{Resultados y especificaciones finales}
Explicitar el grado de cumplimiento de las metas y objetivos planteados inicialmente.  
Determinar especificaciones técnicas como consumo, autonomía, precisión, velocidad, alcance, rangos de trabajo según corresponda a la aplicación.


\subsection{Caracterización del Sistema}

Se realizaron pruebas exhaustivas para caracterizar el rendimiento del sistema:

\subsubsection{Precisión de Posicionamiento}

\begin{table}[H]
\centering
\caption{Pruebas de precisión de posicionamiento}
\label{tab:precision}
\begin{tabular}{|c|c|c|c|}
\hline
\textbf{Eje} & \textbf{Distancia Programada (mm)} & \textbf{Distancia Real (mm)} & \textbf{Error (\%)} \\
\hline
X & 10.000 & 10.002 & 0.02 \\
X & 50.000 & 49.998 & -0.004 \\
X & 100.000 & 100.005 & 0.005 \\
\hline
Y & 10.000 & 10.001 & 0.01 \\
Y & 50.000 & 50.003 & 0.006 \\
Y & 100.000 & 99.997 & -0.003 \\
\hline
Z & 5.000 & 5.001 & 0.02 \\
Z & 20.000 & 19.999 & -0.005 \\
Z & 40.000 & 40.002 & 0.005 \\
\hline
\end{tabular}
\end{table}

\subsubsection{Velocidades de Trabajo}

Las velocidades máximas alcanzadas por el sistema son:
\begin{itemize}
    \item Velocidad de desplazamiento rápido: 1500 mm/min
    \item Velocidad de corte: 300-800 mm/min
    \item Velocidad de grabado: 150-400 mm/min
\end{itemize}

\subsection{Pruebas Funcionales}

\subsubsection{Prueba de Corte Lineal}

Se realizó una prueba de corte lineal en madera MDF de 3mm de espesor. Los resultados mostraron:
\begin{itemize}
    \item Precisión dimensional: ±0.05mm
    \item Acabado superficial: satisfactorio
    \item Repetibilidad: 99.8\%
\end{itemize}

\subsubsection{Prueba de Interpolación Circular}

Se ejecutó un programa G-code para corte circular:

% \begin{lstlisting}[caption={Código G-code para prueba circular}]
% G90         ; Coordenadas absolutas
% G00 X0 Y0   ; Ir al origen
% G01 Z-2 F100 ; Bajar herramienta
% G02 X10 Y0 I5 J0 F300 ; Círculo horario
% G00 Z5      ; Subir herramienta
% M30         ; Fin del programa
% \end{lstlisting}

\subsection{Análisis de Rendimiento}

\subsubsection{Consumo de Potencia}

\begin{table}[H]
\centering
\caption{Análisis de consumo energético}
\label{tab:consumo}
\begin{tabular}{|l|c|c|}
\hline
\textbf{Componente} & \textbf{Corriente (A)} & \textbf{Potencia (W)} \\
\hline
STM32F103C8T6 & 0.02 & 0.1 \\
Motores paso a paso (3x) & 1.2 & 14.4 \\
Drivers A4988 (3x) & 0.15 & 1.8 \\
Electrónica auxiliar & 0.08 & 1.0 \\
\hline
\textbf{Total} & \textbf{1.45} & \textbf{17.3} \\
\hline
\end{tabular}
\end{table}

% Conclusiones
\section{Conclusiones. Ensayo de ingeniería de producto}
Reflexión sobre el desarrollo realizado y perspectivas de mejoras del prototipo.  
Ejercicio de selección de componentes para su paso a un producto comercial o sistema de aplicación industrial o del ámbito que corresponda.

\subsection{Cumplimiento de Objetivos}

El proyecto cumplió satisfactoriamente con todos los objetivos planteados:

\begin{itemize}
    \item Se desarrolló un sistema CNC funcional basado en STM32F103C8T6
    \item El firmware adaptado de GRBL permite interpretación completa de G-code
    \item La precisión alcanzada es adecuada para las aplicaciones objetivo
    \item El costo total del sistema es significativamente menor a alternativas comerciales
\end{itemize}

\subsection{Aprendizajes Obtenidos}

Durante el desarrollo del proyecto se adquirieron conocimientos en:
\begin{itemize}
    \item Programación de microcontroladores ARM Cortex-M3
    \item Control de motores paso a paso y sistemas de movimiento
    \item Interpretación y procesamiento de código G-code
    \item Integración de hardware y software en sistemas embebidos
    \item Técnicas de debugging y optimización en sistemas de tiempo real
\end{itemize}

\subsection{Trabajo Futuro}

Las siguientes mejoras podrían implementarse en versiones futuras:

\subsubsection{Hardware}
\begin{itemize}
    \item Implementación de encoders para retroalimentación de posición
    \item Adición de sensores de temperatura y corriente
    \item Upgrade a drivers de motor más avanzados (TMC2209, TMC5160)
    \item Sistema de refrigeración para cortes prolongados
\end{itemize}

\subsubsection{Software}
\begin{itemize}
    \item Implementación de compensación de backlash
    \item Algoritmos de aceleración adaptativa
    \item Interface web para control remoto
    \item Sistema de detección de colisiones
\end{itemize}

\subsubsection{Mecánica}
\begin{itemize}
    \item Estructura más rígida con perfiles de aluminio extruido
    \item Sistema de transmisión por tornillos de bolas
    \item Aumento del área de trabajo a 400x400x100mm
    \item Sistema de sujeción de piezas mejorado
\end{itemize}

% Referencias
\section{Referencias}
Bibliografía, hojas de datos, guías, enlaces a sitios o documentos de internet.

\begin{enumerate}
  \item \label{ref1} STMicroelectronics. STM32F103C8 Datasheet. \\
  \url{https://www.st.com/en/microcontrollers-microprocessors/stm32f103c8.html}
  
  \item \label{ref2} Pololu Corporation. A4988 Stepper Motor Driver Carrier. \\
  \url{https://www.pololu.com/product/1182}
  
  \item \label{ref3} OMC StepperOnline. Motor NEMA 17HS3430 Datasheet. \\
  \url{https://www.alldatasheet.es/datasheet-pdf/download/1137270/MOTIONKING/17HS3430.html}
  
  \item \label{ref4} OMC StepperOnline. Motor NEMA 17HS8401 Datasheet. \\
  \url{https://www.alldatasheet.es/datasheet-pdf/download/1137270/MOTIONKING/17HS3430.html}
  
  \item \label{ref5} RepRap Community. RAMPS 1.4 Documentation. \\
  \url{https://reprap.org/wiki/RAMPS_1.4}
  
  \item \label{ref6} Arduino STM32 Community. Blue Pill Pinout Reference. \\
  \url{https://github.com/rogerclarkmelbourne/Arduino_STM32/wiki/Blue-Pill-Pinout}
  
  \item \label{ref7} RepRap Community. RAMPS 1.4 Schematic. \\
  \url{https://reprap.org/mediawiki/images/c/c4/RAMPS1.4schematic.png}
\end{enumerate}


\begin{enumerate}
    \item STMicroelectronics. (2023). \textit{STM32F103C8T6 Datasheet}. ST Microelectronics.
    
    \item Simen Svale Skogsrud. (2011). \textit{GRBL: An open source, embedded, high performance g-code-parser and CNC milling controller}. GitHub Repository.
    
    \item ARM Limited. (2023). \textit{ARM Cortex-M3 Technical Reference Manual}. ARM Holdings.
    
    \item Smid, P. (2019). \textit{CNC Programming Handbook: A Comprehensive Guide to Practical CNC Programming}. Industrial Press.
    
    \item Lynch, M. (2018). \textit{CNC Machining Handbook: Building, Programming, and Implementation}. McGraw-Hill Education.
    
    \item Valentino, J., \& Goldenberg, J. (2020). \textit{Introduction to Computer Numerical Control}. Pearson.
    
    \item Texas Instruments. (2022). \textit{Stepper Motor Control Guide}. TI Application Notes.
    
    \item Allegro MicroSystems. (2023). \textit{A4988 Microstepping Driver Datasheet}. Allegro MicroSystems.
\end{enumerate}

% Anexos
\section{Anexos}
Aquellos detalles no disponibles en las referencias de acceso público.

\begin{itemize}
  \item \textbf{Mapa de pines Blue Pill STM32F103C8T6}: Ver referencia \hyperref[ref6]{[6]}
  \item \textbf{Esquema RAMPS 1.4}: Ver referencia \hyperref[ref7]{[7]}
\end{itemize}


\subsection{Anexo A: Esquemático de la PCB}

\begin{figure}[H]
    \centering
    % \includegraphics[width=1.0\textwidth]{esquematico.png}
    \caption{Esquemático completo del sistema}
    \label{fig:esquematico}
\end{figure}

\subsection{Anexo B: Lista de Materiales (BOM)}

\begin{longtable}{|c|p{6cm}|c|c|c|}
\caption{Lista de materiales completa} \label{tab:bom} \\
\hline
\textbf{Ítem} & \textbf{Descripción} & \textbf{Cant.} & \textbf{Precio Unit.} & \textbf{Total} \\
\hline
\endfirsthead

\multicolumn{5}{c}%
{{\bfseries \tablename\ \thetable{} -- continuación de la página anterior}} \\
\hline
\textbf{Ítem} & \textbf{Descripción} & \textbf{Cant.} & \textbf{Precio Unit.} & \textbf{Total} \\
\hline
\endhead

\hline \multicolumn{5}{|r|}{{Continúa en la siguiente página}} \\ \hline
\endfoot

\hline
\endlastfoot

1 & STM32F103C8T6 Blue Pill & 1 & \$5.00 & \$5.00 \\
2 & Driver A4988 & 3 & \$2.50 & \$7.50 \\
3 & Motor NEMA 17 & 3 & \$15.00 & \$45.00 \\
4 & Fuente 12V 5A & 1 & \$12.00 & \$12.00 \\
5 & Perfil de aluminio 20x20 & 2m & \$8.00 & \$16.00 \\
6 & Varilla roscada M8 & 3 & \$3.00 & \$9.00 \\
7 & Rodamientos lineales & 6 & \$2.00 & \$12.00 \\
8 & Correas GT2 & 3m & \$1.50 & \$4.50 \\
9 & Poleas GT2 20 dientes & 3 & \$2.00 & \$6.00 \\
10 & Finales de carrera & 6 & \$1.00 & \$6.00 \\
11 & Cables y conectores & - & \$15.00 & \$15.00 \\
12 & PCB personalizada & 1 & \$10.00 & \$10.00 \\
13 & Componentes electrónicos & - & \$20.00 & \$20.00 \\
14 & Tornillería y fijaciones & - & \$25.00 & \$25.00 \\
15 & Estructura mecánica & - & \$30.00 & \$30.00 \\
\hline
& \textbf{TOTAL} & & & \textbf{\$223.00} \\
\hline
\end{longtable}

\subsection{Anexo C: Código Fuente Principal}

El código fuente completo del proyecto está disponible en el repositorio:
\url{https://github.com/BPera1111/Proyecto_Micro}

\subsubsection{Archivos Principales:}
\begin{itemize}
    \item \texttt{main.c}: Función principal y configuración del sistema
    \item \texttt{grbl/}: Directorio con el firmware GRBL adaptado
    \item \texttt{gcode.py}: Interfaz de usuario en Python
    \item \texttt{examples/}: Archivos G-code de ejemplo
\end{itemize}

\subsection{Anexo D: Resultados de Pruebas Detallados}

\subsubsection{Log de Prueba de Precisión}
\begin{lstlisting}[caption={Extracto del log de pruebas}]
Test Date: 2025-08-11
Test Type: Precision Positioning

X-Axis Test:
Command: G01 X10 F500
Expected: 10.000mm
Measured: 10.002mm
Error: 0.02%

Y-Axis Test:
Command: G01 Y10 F500
Expected: 10.000mm
Measured: 10.001mm
Error: 0.01%

Z-Axis Test:
Command: G01 Z5 F200
Expected: 5.000mm
Measured: 5.001mm
Error: 0.02%
\end{lstlisting}


\section{Marco Teórico}

\subsection{Control Numérico Computarizado (CNC)}

El Control Numérico Computarizado es un método de automatización de máquinas herramientas mediante el uso de software y datos numéricos codificados. Los sistemas CNC ejecutan secuencias predeterminadas de comandos de máquina con poca o ninguna intervención del operador.

\subsection{Microcontrolador STM32F103C8T6}

El STM32F103C8T6 es un microcontrolador de 32 bits basado en el núcleo ARM Cortex-M3 con las siguientes características principales:

\begin{itemize}
    \item Frecuencia de operación de hasta 72 MHz
    \item 64 KB de memoria Flash
    \item 20 KB de memoria RAM
    \item Múltiples periféricos incluyendo UART, SPI, I2C, ADC, PWM
    \item 37 pines GPIO configurables
\end{itemize}

\subsection{Protocolo G-code}

G-code es un lenguaje de programación numérica utilizado principalmente para controlar máquinas herramientas automatizadas. Cada línea de código G-code contiene comandos que especifican movimientos, velocidades, y operaciones de la herramienta.

\subsubsection{Comandos G-code Principales}
\begin{itemize}
    \item \texttt{G00}: Movimiento rápido
    \item \texttt{G01}: Interpolación lineal
    \item \texttt{G02/G03}: Interpolación circular
    \item \texttt{M03/M05}: Control del husillo
    \item \texttt{G90/G91}: Coordenadas absolutas/incrementales
\end{itemize}

\subsection{GRBL}

GRBL es un firmware de código abierto que transforma comandos G-code en señales de control para motores paso a paso. Originalmente desarrollado para microcontroladores AVR, ha sido portado a múltiples plataformas incluyendo ARM.

\section{Metodología}

\subsection{Diseño del Sistema}

El desarrollo del proyecto se dividió en las siguientes etapas:

\begin{enumerate}
    \item Análisis de requerimientos y especificaciones técnicas
    \item Diseño del hardware y selección de componentes
    \item Implementación del firmware basado en GRBL
    \item Desarrollo de la interfaz de usuario
    \item Integración y pruebas del sistema
    \item Validación y caracterización de la máquina
\end{enumerate}

\subsection{Herramientas Utilizadas}

\subsubsection{Hardware}
\begin{itemize}
    \item Microcontrolador STM32F103C8T6 (Blue Pill)
    \item Drivers de motores paso a paso (A4988)
    \item Motores paso a paso NEMA 17
    \item Estructura mecánica de aluminio
    \item Fuente de alimentación conmutada
\end{itemize}

\subsubsection{Software}
\begin{itemize}
    \item STM32CubeIDE para desarrollo del firmware
    \item Python para la interfaz de usuario
    \item GRBL como base del firmware de control
    \item CAM software para generación de G-code
\end{itemize}

\section{Desarrollo}

\subsection{Diseño del Hardware}

\subsubsection{Arquitectura del Sistema}

El sistema CNC está compuesto por los siguientes módulos principales:

\begin{figure}[H]
    \centering
    % \includegraphics[width=0.8\textwidth]{diagrama_bloques.png}
    \caption{Diagrama de bloques del sistema CNC}
    \label{fig:diagrama_bloques}
\end{figure}

\subsubsection{Selección de Componentes}

La selección de componentes se basó en los siguientes criterios:
\begin{itemize}
    \item Compatibilidad con el microcontrolador STM32
    \item Disponibilidad comercial y costo
    \item Especificaciones técnicas apropiadas para la aplicación
    \item Facilidad de integración y mantenimiento
\end{itemize}

\subsubsection{Diseño de la PCB}

Se diseñó una PCB personalizada que integra:
\begin{itemize}
    \item Conectores para el microcontrolador STM32
    \item Circuitos de acondicionamiento de señales
    \item Conectores para drivers de motores
    \item Circuitos de protección y filtrado
    \item Interfaces de comunicación
\end{itemize}

\subsection{Implementación del Firmware}

\subsubsection{Adaptación de GRBL}

El firmware GRBL fue adaptado para trabajar con el microcontrolador STM32F103C8T6. Las principales modificaciones incluyeron:

\begin{itemize}
    \item Configuración de periféricos específicos del STM32
    \item Adaptación de las rutinas de temporización
    \item Implementación de drivers para comunicación USB
    \item Optimización para el núcleo ARM Cortex-M3
\end{itemize}

\subsubsection{Estructura del Firmware}

% \begin{lstlisting}[language=C, caption={Estructura principal del firmware}]
% int main(void) {
%     // Inicialización del sistema
%     HAL_Init();
%     SystemClock_Config();
    
%     // Inicialización de periféricos
%     MX_GPIO_Init();
%     MX_USB_DEVICE_Init();
%     MX_TIM1_Init();
    
%     // Inicialización de GRBL
%     grbl_init();
    
%     // Bucle principal
%     while (1) {
%         grbl_process();
%     }
% }
% \end{lstlisting}

\subsubsection{Configuración de Motores}

% \begin{lstlisting}[language=C, caption={Configuración de pines para control de motores}]
% // Configuración de pines para eje X
% #define X_STEP_PIN    GPIO_PIN_2
% #define X_DIR_PIN     GPIO_PIN_3
% #define X_ENABLE_PIN  GPIO_PIN_4

% // Configuración de pines para eje Y
% #define Y_STEP_PIN    GPIO_PIN_5
% #define Y_DIR_PIN     GPIO_PIN_6
% #define Y_ENABLE_PIN  GPIO_PIN_7

% // Configuración de pines para eje Z
% #define Z_STEP_PIN    GPIO_PIN_8
% #define Z_DIR_PIN     GPIO_PIN_9
% #define Z_ENABLE_PIN  GPIO_PIN_10
% \end{lstlisting}

\subsection{Interfaz de Usuario}

Se desarrolló una interfaz de usuario en Python que permite:

\begin{itemize}
    \item Cargar archivos G-code
    \item Visualizar el toolpath
    \item Controlar manualmente los ejes
    \item Monitorear el estado de la máquina
    \item Configurar parámetros de funcionamiento
\end{itemize}

\begin{lstlisting}[language=Python, caption={Código principal de la interfaz}]
import tkinter as tk
from tkinter import filedialog, messagebox
import serial
import threading

class CNCController:
    def __init__(self):
        self.serial_connection = None
        self.setup_gui()
    
    def setup_gui(self):
        self.root = tk.Tk()
        self.root.title("Control CNC STM32")
        self.create_widgets()
    
    def load_gcode_file(self):
        filename = filedialog.askopenfilename(
            filetypes=[("G-code files", "*.gcode"), ("All files", "*.*")]
        )
        if filename:
            self.process_gcode_file(filename)
\end{lstlisting}

\section{Análisis y Discusión}

\subsection{Ventajas del Sistema Desarrollado}

\begin{itemize}
    \item \textbf{Costo reducido:} Utilización de componentes de bajo costo y disponibles comercialmente
    \item \textbf{Flexibilidad:} Firmware basado en estándares abiertos permite modificaciones
    \item \textbf{Precisión adecuada:} Errores de posicionamiento menores al 0.02\%
    \item \textbf{Facilidad de uso:} Interfaz intuitiva para operadores sin experiencia previa
\end{itemize}

\subsection{Limitaciones Identificadas}

\begin{itemize}
    \item \textbf{Área de trabajo limitada:} 200x200x50mm puede ser restrictivo para algunas aplicaciones
    \item \textbf{Velocidades de corte:} Menores comparadas con sistemas comerciales
    \item \textbf{Tipos de material:} Limitado a materiales blandos debido al torque de los motores
\end{itemize}

\subsection{Comparación con Sistemas Comerciales}

\begin{table}[H]
\centering
\caption{Comparación con sistemas comerciales}
\label{tab:comparacion}
\begin{tabular}{|l|c|c|c|}
\hline
\textbf{Característica} & \textbf{Sistema Desarrollado} & \textbf{CNC Comercial Básica} & \textbf{CNC Comercial Avanzada} \\
\hline
Precisión & ±0.05mm & ±0.02mm & ±0.005mm \\
Velocidad máx. & 1500 mm/min & 3000 mm/min & 15000 mm/min \\
Área de trabajo & 200x200x50mm & 300x300x80mm & 600x400x150mm \\
Costo aprox. & \$200 & \$800 & \$5000+ \\
\hline
\end{tabular}
\end{table}

\section{Aplicaciones Prácticas}

\subsection{Casos de Uso Implementados}

\subsubsection{Grabado de PCBs}

El sistema demostró capacidad para realizar grabado de circuitos impresos con las siguientes especificaciones:
\begin{itemize}
    \item Ancho mínimo de pista: 0.2mm
    \item Precisión de via: ±0.05mm
    \item Tiempo de grabado: 15-30 min para PCB de 50x50mm
\end{itemize}

\subsubsection{Corte de Materiales}

Se realizaron cortes exitosos en:
\begin{itemize}
    \item MDF hasta 5mm de espesor
    \item Acrílico hasta 3mm de espesor
    \item Cartón y papel de cualquier grosor
    \item Espuma de poliestireno hasta 20mm
\end{itemize}

\subsection{Proyectos Educativos}

El sistema se utilizó para:
\begin{itemize}
    \item Fabricación de maquetas arquitectónicas
    \item Creación de plantillas y moldes
    \item Proyectos de arte y diseño
    \item Prototipado rápido de piezas mecánicas simples
\end{itemize}


\end{document}
